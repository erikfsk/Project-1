\documentclass[10pt,showpacs,preprintnumbers,footinbib,amsmath,amssymb,aps,prl,twocolumn,groupedaddress,superscriptaddress,showkeys]{revtex4-1}
\usepackage{graphicx}
\usepackage{dcolumn}
\usepackage{bm}
\usepackage[colorlinks=true,urlcolor=blue,citecolor=blue]{hyperref}
\usepackage{color}
\usepackage{listings}


\begin{document}



\title{Life  on Devaron, a forest planet with an ancient Jedi Temple.}

\author{S.~O.~Mebody}
\affiliation{Department of Procrastination, University of Akiva}
\author{A.~N.~Ybody} 
\affiliation{Department of Odd Travels, University of Anoat}
\author{N.~O.~Body}
\affiliation{Department of Even Travels, University of Ambria}

\begin{abstract}
We present the first calculations for rodlike microswimmers from outer space.
\end{abstract}



\maketitle



\section{Introduction}
The aim of this project is to get familiar with various vector and matrix operations,
from dynamic memory allocation to the usage of programs in the library
package of the course. 
For Fortran users memory handling and most matrix and vector operations
are included in the ANSI standard of Fortran 90/95. Array handling in Python is also rather trivial. For C++ user however,
there are several possible options. Two are listed here.

\begin{itemize}
  \item For this exercise we recommend that you make your own functions for dynamic memory allocation of a  vector and a matrix. You don't need to write a class for this operations.  Use then the  library package lib.cpp with its header file  lib.hpp for obtaining LU-decomposed matrices, solve linear equations etc.

  \item A very good and often recommended library for C++ handling of arrays is the library Armadillo, to be found at \url{arma.sourceforge.net}.  We will discuss the usage of this library during the lab sessions and lectures. Armadillo has also an interface to Lapack functions for solving systems of linear equations.
\end{itemize}

\noindent
Your program, whether it is written in C++, Python 
or Fortran2008, should include
dynamic memory handling of matrices and vectors. 


\section{Methods and algortihms}


Many important differential equations in  Science can be written as 
linear second-order differential equations

\begin{equation*}
\frac{d^2y}{dx^2}+k^2(x)y = f(x),
\end{equation*}
where $f$ is normally called the inhomogeneous term and $k^2$ is a real function.

A classical equation from electromagnetism is Poisson's equation.
The electrostatic potential $\Phi$ is generated by a localized charge
distribution $\rho (\mathbf{r})$.   In three dimensions 
it reads

\begin{equation*}
\nabla^2 \Phi = -4\pi \rho (\mathbf{r}).
\end{equation*}
With a spherically symmetric $\Phi$ and $\rho (\mathbf{r})$  the equations
simplifies to a one-dimensional equation in $r$, namely

\begin{equation*}
\frac{1}{r^2}\frac{d}{dr}\left(r^2\frac{d\Phi}{dr}\right) = -4\pi \rho(r),
\end{equation*}
which can be rewritten via a substitution $\Phi(r)= \phi(r)/r$ as

\begin{equation*}
\frac{d^2\phi}{dr^2}= -4\pi r\rho(r).
\end{equation*}
The inhomogeneous term $f$ or source term is given by the charge distribution
$\rho$  multiplied by $r$ and the constant $-4\pi$.

We will rewrite this equation by letting $\phi\rightarrow u$ and 
$r\rightarrow x$. 
The general one-dimensional Poisson equation reads then

\begin{equation*}
-u''(x) = f(x).
\end{equation*}


In this project we will solve the one-dimensional Poisson equation
with Dirichlet boundary conditions by rewriting it as a set of linear equations.


To be more explicit we will solve the equation

\begin{equation*}
-u''(x) = f(x), \hspace{0.5cm} x\in(0,1), \hspace{0.5cm} u(0) = u(1) = 0.
\end{equation*}
and we define the discretized approximation  to $u$ as $v_i$  with 
grid points $x_i=ih$   in the interval from $x_0=0$ to $x_{n+1}=1$.
The step length or spacing is defined as $h=1/(n+1)$. 
We have then the boundary conditions $v_0 = v_{n+1} = 0$.
We  approximate the second
derivative of $u$ with

\begin{equation*}
   -\frac{v_{i+1}+v_{i-1}-2v_i}{h^2} = f_i  \hspace{0.5cm} \mathrm{for} \hspace{0.1cm} i=1,\dots, n,
\end{equation*}
where $f_i=f(x_i)$.
We  can rewrite this equation as a linear set of equations of the form

\begin{equation*}
   \mathbf{A}\mathbf{v} = \tilde{\mathbf{b}},
\end{equation*}
where $\mathbf{A}$ is an $n\times n$  tridiagonal matrix which we rewrite as

\[
    \mathbf{A} = \begin{bmatrix}
                           2& -1& 0 &\dots   & \dots &0 \\
                           -1 & 2 & -1 &0 &\dots &\dots \\
                           0&-1 &2 & -1 & 0 & \dots \\
                           & \dots   & \dots &\dots   &\dots & \dots \\
                           0&\dots   &  &-1 &2& -1 \\
                           0&\dots    &  & 0  &-1 & 2 \\
                      \end{bmatrix},
\]
and $\tilde{b}_i=h^2f_i$.


In our case we will assume  that the source term is 
$f(x) = 100e^{-10x}$, and keep the same interval and boundary 
conditions. Then the above differential equation
has a closed-form  solution given by $u(x) = 1-(1-e^{-10})x-e^{-10x}$ (convince yourself that this is correct by inserting the
solution in the Poisson equation).  We will compare
our numerical solution with this result in the next exercise. 


We can rewrite our matrix $\mathbf{A}$ in terms of one-dimensional vectors $a,b,c$  
of length $1:n$. 
Our linear equation reads

\[
    \mathbf{A} = \begin{bmatrix}
                           b_1& c_1 & 0 &\dots   & \dots &\dots \\
                           a_1 & b_2 & c_2 &\dots &\dots &\dots \\
                           & a_2 & b_3 & c_3 & \dots & \dots \\
                           & \dots   & \dots &\dots   &\dots & \dots \\
                           &   &  &a_{n-2}  &b_{n-1}& c_{n-1} \\
                           &    &  &   &a_{n-1} & b_n \\
                      \end{bmatrix}\begin{bmatrix}
                           v_1\\
                           v_2\\
                           \dots \\
                          \dots  \\
                          \dots \\
                           v_n\\
                      \end{bmatrix}
  =\begin{bmatrix}
                           \tilde{b}_1\\
                           \tilde{b}_2\\
                           \dots \\
                           \dots \\
                          \dots \\
                           \tilde{b}_n\\
                      \end{bmatrix}.
\]
We can compute the relative error  in the data set $i=1,\dots, n$,by setting up

\[
   \epsilon_i=log_{10}\left(\left|\frac{v_i-u_i}
                 {u_i}\right|\right),
\]
as function of $log_{10}(h)$ for the function values $u_i$ and $v_i$.
For each step length extract the max value of the relative error.  
Try to increase $n$ to $n=10^7$.  Make a table of the results and 
comment your results. You can use either the algorithm from b) or c). 

To compute the elapsed time in c++ you can use the following statements
\begin{lstlisting}
...
#include "time.h"   //  you have to include the time.h header
int main()
{
    // declarations of variables 
    ...
    clock_t start, finish;  //  declare start and final time
    start = clock();
    // your code is here, do something and then get final time
    finish = clock();
    ( (finish - start)/CLOCKS_PER_SEC );
...
\end{lstlisting}

\section{Our results}

We present our results in Fig.~\ref{fig:figure1}.

\begin{figure} \label{fig:figure1}
%\includegraphics[scale=0.4]{figure1.pdf}
\caption{Our results could not be better}
\end{figure}

\section{Conclusions and perspectives}
What a wonderful world!

\begin{thebibliography}{99}
\bibitem{heisenberg} W. Heisenberg, Zeits. f. Physik {\bf 77}, 1 (1932).
\bibitem{miller2006} G.~A.~Miller, A.~K.~Opper, and E.~J.~Stephenson, Annu.~Rev.~Nucl.~Sci.~{\bf 56}, 253 (2006).
\bibitem{ekstrom2015} A.~Ekstr\"om, G.~R.~Jansen, K.~A.~Wendt, G.~Hagen, T.~Papenbrock, B.~D.~Carlsson, C.~Forssen, M.~Hjorth-Jensen, P.~Navratil, and W.~Nazarewicz, Phys.~Rev.~C {\bf 91}, 051301(R) (2015).
\bibitem{brown1977} B. A. Brown, A. Arima and J. B. McGrory, Nucl. Phys. {\bf A277}, 77 (1977) and references therein.
\end{thebibliography}

\end{document}



